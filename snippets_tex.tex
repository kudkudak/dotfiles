% Arxiving paper

1. Add colorlinks to hyperref
2. Make sure all eps are converted to pdf (without name change)

% Removing unsued figures

# small script that deletes unused graphic files in a latex project.
# specify graphic directory and logfile. Unused files will be deleted in specified directory.

import os

directory = 'figs'
logfile = 'main.log'
enc = 'Latin-1' # For Overleaf, try 'Latin-1' if issues encountered...

for filename in os.listdir(directory):
    if filename in open(logfile).read():
        print(filename + ' in use.')
    else:
        if os.path.isfile(os.path.join(directory, filename)):
            print(filename + ' not in use - deleting.')
            os.remove(os.path.join(directory, filename))
        else:
            print(filename + ' is a directory.')

% Subfigures

\begin{figure}
\centering
\begin{subfigure}[t]{0.49\linewidth}
\includegraphics[height=2.0cm,trim=0.in 0.in 0.in 0.in,clip]{figs/spectrum_simplecnn_acc.pdf}
\end{subfigure}%
\begin{subfigure}[t]{0.49\linewidth}
\includegraphics[height=2.0cm,trim=0.in 0.in 0.in 0.in,clip]{figs/spectrum_resnet32_acc.pdf}
\caption{Resnet-32}
\end{subfigure}
\begin{subfigure}[t]{0.49\linewidth}
\includegraphics[height=2.0cm,trim=0.in 0.in 0.in 0.in,clip]{figs/spectrum_zoom_simplecnn_acc.pdf}
 \caption{SimpleCNN (zoom)}
\end{subfigure}%
\begin{subfigure}[t]{0.49\linewidth}
\includegraphics[height=2.0cm,trim=0.in 0.in 0.in 0.in,clip]{figs/spectrum_zoom_resnet32_acc.pdf}
\caption{Resnet-32 (zoom)}
\end{subfigure}
\caption{\textbf{Top}: Evolution of the top $10$ eigenvalues of the Hessian for SimpleCNN and Resnet-32 trained on the CIFAR-10 dataset with $\eta=0.1$ and $S=128$. \textbf{Bottom}: Zoom on the evolution of the top $10$ eigenvalues in the selected range of iterations in the beginning of training. Oscillatory-like evolution and a sharp growth of the largest eigenvalues are visible. Training and test accuracy of the corresponding models are provided for reference. Resnet-32  and SimpleCNN achieve $88\%$ and $86\%$ test accuracy, respectively.}
\label{fig:largesteigenvalues}
\end{figure}
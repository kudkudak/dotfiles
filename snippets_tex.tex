% Typical preamble

\usepackage[colorlinks=True]{hyperref}
\usepackage[T1]{fontenc}
\usepackage[polish]{babel}
\usepackage[utf8]{inputenc}
\usepackage{natbib}
\usepackage{multirow}
\usepackage{placeins}
\usepackage{graphicx}
\usepackage{subfigure}
\usepackage{url}
\usepackage{amsthm}
\usepackage{float}
\usepackage{amsmath}

% Minipage and hspace
    \begin{minipage}{.48\textwidth}
    \includegraphics[width=0.47\textwidth]{fig/1809nobnscnnc10sweeplrG_SN.pdf}%
    \includegraphics[width=0.47\textwidth]{fig/1809nobnscnnc10sweeplrconditioningratio.pdf}
    \hspace*{0.3cm}
     \includegraphics[width=0.9\textwidth]{fig/1809nobnscnnc10sweeplrlegend.pdf}
    \end{minipage}

% Diagramy

http://latex-cookbook.net/articles/smart-diagrams/

% Side by side 

\begin{frame}{Research Question}
% Czemu w tym doktoracie zajmuje sie sieciami neuronowymi
\begin{figure}
\centering
\begin{subfigure}[t]{0.45\textwidth}
  \centering
   \includegraphics<1,2>[width=0.4\paperwidth]{fig/optim_in_DL.png} 
 \caption{Choice of the optimization method change generalization. An important difference between "deep learning" and "shallow" learning. \textcolor{red}{Why?}}
\end{subfigure} 
~
\pause
\begin{subfigure}[t]{0.45\textwidth}
  \centering
%\includegraphics<2>[width=0.4\paperwidth]{fig/mtla.png}
\includegraphics<2>[width=0.4\paperwidth]{fig/mtlb.png}
\caption{Mult-task learning is both key for real-life application s of DL, and extremely challenging in practice. This and other open challenges in DL might be due to optimization.}
%\caption{Wykrywanie raka piersi za pomocą sieci neuronowej. Działa świetnie, ale (według mnie) za mało rozumiemy sieci neuronowy aby używać tego typu modele klinicznie.}
\end{subfigure}
\end{figure}

% Remove unused

import glob
import os

TEX=open("../main.tex").read()

SRC1="/Users/jastrs01/Dropbox/Apps/results_experiments/gpfs/data/geraslab/jastrs01/code/magic_sgd/experiments/07_19_base_experiment/plotting"
SRC2="/Users/jastrs01/Dropbox/Apps/results_experiments/gpfs/data/geraslab/jastrs01/code/magic_sgd/experiments/07_19_base_experiment/misc_plotting"
SRC3="/Users/jastrs01/Dropbox/Apps/results_experiments/gpfs/data/geraslab/jastrs01/code/magic_sgd/experiments/07_19_base_experiment/eyecandy"
for SRC in [SRC1, SRC2, SRC3]:
    for f in glob.glob("*pdf"):
        if os.path.basename(f) not in TEX:
            print("WARNING! {} not found in main.tex".format(f))
        else:
            print("OK")
        a = os.path.join(SRC, os.path.basename(f))
        os.system("cp {} {}".format(a,f))

% Update figures

%% Workflow for adding new figures is to copy manually and use in overleaf

import glob
import os
SRC="$HOME/Dropbox/Apps/results_experiments/magic_sgd/experiments/07_19_base_experiment/plotting"
SRC1="/Users/jastrs01/Dropbox/Apps/results_experiments/gpfs/data/geraslab/jastrs01/code/magic_sgd/experiments/07_19_base_experiment/plotting"
SRC2="/Users/jastrs01/Dropbox/Apps/results_experiments/gpfs/data/geraslab/jastrs01/code/magic_sgd/experiments/07_19_base_experiment/misc_plotting"
for SRC in [SRC1, SRC2]:
   for f in glob.glob("*pdf"):
    a=os.path.join(SRC, os.path.basename(f))
    os.system("cp {} {}".format(a,f))

% GIT and overleaf

git clone https://git.overleaf.com/5d49b8e7c717ee6322f44c55 paper
Login with your credentials (if using Google account in Overleaf then login using random credentials and follow the instructions)

% Arxiving paper

0. pdflatex bibtex to make sure bbl fire exists
1. Add colorlinks to hyperref (sometimes remove hyperref - arxiv preloads it!)
2. Make sure all eps are converted to pdf (without name change - just use generated by pdflatex), remove any
reference to eps! (e.g. improtant for On the relation paper)
3. main.bib and main.bbl in the same folder
4. no other tex files 

\definecolor{darkblue}{rgb}{0.0,0.0,0.55}
\hypersetup{
  colorlinks = true,
  citecolor  = darkblue,
  linkcolor  = darkblue,
  citecolor  = darkblue,
  filecolor  = darkblue,
  urlcolor   = darkblue,
}
\usepackage{color}

% Removing unsued figures

# small script that deletes unused graphic files in a latex project.
# specify graphic directory and logfile. Unused files will be deleted in specified directory.

import os

directory = 'figs'
logfile = 'main.log'
enc = 'Latin-1' # For Overleaf, try 'Latin-1' if issues encountered...

for filename in os.listdir(directory):
    if filename in open(logfile).read():
        print(filename + ' in use.')
    else:
        if os.path.isfile(os.path.join(directory, filename)):
            print(filename + ' not in use - deleting.')
            os.remove(os.path.join(directory, filename))
        else:
            print(filename + ' is a directory.')

% Absolute position beamer

\begin{tikzpicture}[remember picture,overlay]
  \node[anchor=north east,inner sep=0pt] at ($(current page.north east)+(2cm,5cm)$) {
     \includegraphics[width=0.2\paperwidth]{fig/breakeven_a.png}
  };
\end{tikzpicture}


% Columns beamer

\begin{columns}[T] % align columns
\begin{column}{.25\textwidth}
%\color{red}\rule{\linewidth}{4pt}
\begin{figure}
 \includegraphics[height=1.5cm ,trim=0.in 0.in 0.in 0.in,clip]{fig/photo.jpg}
 \end{figure}
\end{column}%
\hfill
\begin{column}{.48\textwidth}

%\color{blue}\rule{\linewidth}{4pt}
\Large
\href{kudkudak.github.io}{kudkudak.github.io}
\end{column}%
\end{columns}


%% QUOTATION %%
\begin{frame}{Goal of machine learning}
\begin{quote}
\textit{With four parameters I can fit an elephant, and with five I can make him wiggle his trunk}.
\end{quote}
\hspace{8cm} John von Neumann
%John von Neumann
\begin{center}
\includegraphics[width=.4\paperwidth]{fig/elephant.png}
\end{center}
\end{frame}


% Block

% Subfigure beamer

\begin{figure}
\centering
\begin{subfigure}[t]{0.45\textwidth}
  \centering
    \includegraphics[height=4.5cm ,trim=0.in 0.in 0.in 0.in,clip]{paper_figs/same_noise_beta1_beta4.pdf}
  \caption{{\tiny{left: $\beta=1$, right: $\beta=4$}}}
\end{subfigure}
~
\begin{subfigure}[t]{0.45\textwidth}
  \centering
 \includegraphics[height=4.5cm ,trim=0.in 0.in 0.in 0.in,clip]{paper_figs/same_noise_beta1_beta0_25.pdf}
    \caption{{\tiny{left: $\beta=1$, right:$\beta=0.25$}}}
\end{subfigure}
\label{fig:same_noise_diff_bp}
\end{figure}

% Subfigures (USE subcaption package not subfigure!!!)

\begin{figure}
\centering
\begin{subfigure}[t]{0.49\linewidth}
\includegraphics[height=2.0cm,trim=0.in 0.in 0.in 0.in,clip]{figs/spectrum_simplecnn_acc.pdf}
\end{subfigure}%
\begin{subfigure}[t]{0.49\linewidth}
\includegraphics[height=2.0cm,trim=0.in 0.in 0.in 0.in,clip]{figs/spectrum_resnet32_acc.pdf}
\caption{Resnet-32}
\end{subfigure}
\begin{subfigure}[t]{0.49\linewidth}
\includegraphics[height=2.0cm,trim=0.in 0.in 0.in 0.in,clip]{figs/spectrum_zoom_simplecnn_acc.pdf}
 \caption{SimpleCNN (zoom)}
\end{subfigure}%
\begin{subfigure}[t]{0.49\linewidth}
\includegraphics[height=2.0cm,trim=0.in 0.in 0.in 0.in,clip]{figs/spectrum_zoom_resnet32_acc.pdf}
\caption{Resnet-32 (zoom)}
\end{subfigure}
\caption{\textbf{Top}: Evolution of the top $10$ eigenvalues of the Hessian for SimpleCNN and Resnet-32 trained on the CIFAR-10 dataset with $\eta=0.1$ and $S=128$. \textbf{Bottom}: Zoom on the evolution of the top $10$ eigenvalues in the selected range of iterations in the beginning of training. Oscillatory-like evolution and a sharp growth of the largest eigenvalues are visible. Training and test accuracy of the corresponding models are provided for reference. Resnet-32  and SimpleCNN achieve $88\%$ and $86\%$ test accuracy, respectively.}
\label{fig:largesteigenvalues}
\end{figure}

% Boxes and definitions
\newtcolorbox{titlelessblock}{
  enhanced,
  boxsep=0.25ex,
  arc=1.25ex,
  opacityframe=.6,
  opacityback=.6,
  colback=red!70!black,
  colframe=red!70!black,
  boxrule=0pt
}


% Movie
%% NOTE: needs Adobe Reader
brew install ffmpeg
ffmpeg -i G.gif -vf scale="trunc(iw/2)*2:trunc(ih/2)*2" -c:a copy -c:v libx264 -profile:v high -pix_fmt yuv420p -g 2 -r 30 G.mp4
%% NOTE: -g is important
% NOTE: Only mov works on OSX
\begin{frame}{}
\centering    
%\includemovie[
%  repeat,autoplay,autoresume,continue,poster
%]{6cm}{6cm}{fig/G.mp4}
\includemedia[
width=0.6\linewidth,height=0.45\linewidth,
addresource=fig/G.mp4, 
transparent,
activate=pageopen,
flashvars={
source=fig/G.mp4
&autoPlay=true
&autoRewind=false,
&loop=true % loop video
}
]{}{VPlayer.swf}
\end{frame}
% NOTE: Looping depends on ffpmeg. Not sure why
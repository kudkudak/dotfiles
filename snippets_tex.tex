% Arxiving paper

0. pdflatex bibtex to make sure bbl fire exists
1. Add colorlinks to hyperref (sometimes remove hyperref - arxiv preloads it!)
2. Make sure all eps are converted to pdf (without name change - just use generated by pdflatex), remove any
reference to eps! (e.g. improtant for On the relation paper)
3. main.bib and main.bbl in the same folder
4. no other tex files 

% Removing unsued figures

# small script that deletes unused graphic files in a latex project.
# specify graphic directory and logfile. Unused files will be deleted in specified directory.

import os

directory = 'figs'
logfile = 'main.log'
enc = 'Latin-1' # For Overleaf, try 'Latin-1' if issues encountered...

for filename in os.listdir(directory):
    if filename in open(logfile).read():
        print(filename + ' in use.')
    else:
        if os.path.isfile(os.path.join(directory, filename)):
            print(filename + ' not in use - deleting.')
            os.remove(os.path.join(directory, filename))
        else:
            print(filename + ' is a directory.')

% Columns beamer

\begin{columns}[T] % align columns
\begin{column}{.25\textwidth}
%\color{red}\rule{\linewidth}{4pt}
\begin{figure}
 \includegraphics[height=1.5cm ,trim=0.in 0.in 0.in 0.in,clip]{fig/photo.jpg}
 \end{figure}
\end{column}%
\hfill
\begin{column}{.48\textwidth}

%\color{blue}\rule{\linewidth}{4pt}
\Large
\href{kudkudak.github.io}{kudkudak.github.io}
\end{column}%
\end{columns}


%% QUOTATION %%
\begin{frame}{Goal of machine learning}
\begin{quote}
\textit{With four parameters I can fit an elephant, and with five I can make him wiggle his trunk}.
\end{quote}
\hspace{8cm} John von Neumann
%John von Neumann
\begin{center}
\includegraphics[width=.4\paperwidth]{fig/elephant.png}
\end{center}
\end{frame}

% Subfigure beamer

\begin{figure}
\centering
\begin{subfigure}[t]{0.45\textwidth}
  \centering
    \includegraphics[height=4.5cm ,trim=0.in 0.in 0.in 0.in,clip]{paper_figs/same_noise_beta1_beta4.pdf}
  \caption{{\tiny{left: $\beta=1$, right: $\beta=4$}}}
\end{subfigure}
~
\begin{subfigure}[t]{0.45\textwidth}
  \centering
 \includegraphics[height=4.5cm ,trim=0.in 0.in 0.in 0.in,clip]{paper_figs/same_noise_beta1_beta0_25.pdf}
    \caption{{\tiny{left: $\beta=1$, right:$\beta=0.25$}}}
\end{subfigure}
\label{fig:same_noise_diff_bp}
\end{figure}

% Subfigures

\begin{figure}
\centering
\begin{subfigure}[t]{0.49\linewidth}
\includegraphics[height=2.0cm,trim=0.in 0.in 0.in 0.in,clip]{figs/spectrum_simplecnn_acc.pdf}
\end{subfigure}%
\begin{subfigure}[t]{0.49\linewidth}
\includegraphics[height=2.0cm,trim=0.in 0.in 0.in 0.in,clip]{figs/spectrum_resnet32_acc.pdf}
\caption{Resnet-32}
\end{subfigure}
\begin{subfigure}[t]{0.49\linewidth}
\includegraphics[height=2.0cm,trim=0.in 0.in 0.in 0.in,clip]{figs/spectrum_zoom_simplecnn_acc.pdf}
 \caption{SimpleCNN (zoom)}
\end{subfigure}%
\begin{subfigure}[t]{0.49\linewidth}
\includegraphics[height=2.0cm,trim=0.in 0.in 0.in 0.in,clip]{figs/spectrum_zoom_resnet32_acc.pdf}
\caption{Resnet-32 (zoom)}
\end{subfigure}
\caption{\textbf{Top}: Evolution of the top $10$ eigenvalues of the Hessian for SimpleCNN and Resnet-32 trained on the CIFAR-10 dataset with $\eta=0.1$ and $S=128$. \textbf{Bottom}: Zoom on the evolution of the top $10$ eigenvalues in the selected range of iterations in the beginning of training. Oscillatory-like evolution and a sharp growth of the largest eigenvalues are visible. Training and test accuracy of the corresponding models are provided for reference. Resnet-32  and SimpleCNN achieve $88\%$ and $86\%$ test accuracy, respectively.}
\label{fig:largesteigenvalues}
\end{figure}

% Movie
\begin{frame}{Solution \#2: simulation}
\LARGE
\begin{center}
 \movie[%
                height = 7cm,%
                width = 7cm,%
                showcontrols,%
                poster%
            ]%
            {\includegraphics[width=0.5\textwidth]{fig/docking_video.png}}
            {docking.mov}
            
            
            \end{center}
\end{frame}